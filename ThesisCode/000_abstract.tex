\thispagestyle{empty}
%%%%%%%%%%%%%%%%%%%%%%%%%%%%%%%%%%%%%%%%%%%%%%%%%%%%%%%%%%%%%%%%%%

\textbf{Abstract}

\small
Members of rural households in sub-Saharan Africa (SSA) are both vulnerable to the health burdens that stem from food insecurity and central to improving the availability and affordability of wholesome foods. It has been estimated that chronic and hidden hunger can be alleviated by implementing a suite of nutrition-specific interventions at a cost of US\$9.6 billion per annum. This can be accelerated with complementary food system-based interventions. However, such interventions are hampered by a limited understanding of food security status and its associations with rural livelihoods. Therefore, the primary objective of this thesis was to describe, analyse and understand food security in rural landholding households in predominantly mixed crop-livestock agricultural systems of sub-Saharan Africa (SSA). The secondary objective was to improve the methodological basis of household level food security studies.
The rural household multi-indicator survey (RHOMIS) tool was developed to describe and analyse the circumstances of rural households. The RHOMIS tool aims to adhere to the principles of being time-efficient, utilitarian, user-friendly, flexible and reliable. The credibility, consistency and reliability of data collected using three different farm household surveys. The shorter and more targeted survey tool, RHOMIS, performed better in terms of staying within credible bounds. Measurements of maize yields and land area owned were found to be less reliable than other variables. Despite the limitations in data quality, our analysis shows that if the same farm households are followed over time, the sample sizes needed to detect substantial changes are in the order of hundreds of surveys, and not in the thousands.
The RHOMIS tool was then used to quantify changes in livelihoods and food security status in an urban linked, high potential region of Tanzania. Households in the study site adaptively responded to local and national circumstances. Changes in land ownership, livestock-holdings and high value crop production were most likely related to market opportunities and personal circumstances, rather than to direct interventions. Several households made strategic changes by expanding land ownership, planting perennial crops and investing in exotic cattle breeds; many households tactically utilised their land for diversified, mixed crop-livestock production. A central finding of this study is that the complex risk management strategies and market responsiveness demonstrated by the `Rising' clusters are at odds with single focus activities that external organisations tend to promote.
Subsequently, instances of chronic and hidden hunger were analysed in two provinces of Burkina Faso. The results of this study show that in both provinces, the ability to purchase food is what differentiates the more food secure households from their less food secure counterparts. This finding does not detract from the utility of subsistence production -- where consumption of own-farm sourced food catered for between 72\% and 91\% of the annual energy requirements. Further, households were observed to be pursuing market-oriented strategies in combination with production diversification -- likely to reduce risk exposure to climatic or economic shocks.
In a large sample of households across SSA, we found that as many as 40\% of households were classified as chronically hungry in the lean period. Prevalence of micronutrient dietary gaps were high, ranging from 35\% of households to 68\%. Vulnerability to dietary gaps differed by household composition, livelihood characteristics and agro-ecological zone (AEZ). It is the combination of livelihood characteristics and the agro-ecological production potential that drive the availability of food and income. It was found that households fail to purchase food categories that nutritionally complement their own agricultural products. Furthermore, households with a livestock component to their farm were found to have a lower prevalence of chronic and hidden hunger.
In extended analyses, the gender of household head and stage of life were found to be associated with the number of household inhabitants and thus influence nutritional requirements and food security status throughout the year. The high prevalence of food insecurity, the complexity of associations and the failure to nutritionally complement own-production with purchases have implications for developing effective interventions. Programs can be designed as `packages' of agricultural and non-agricultural interventions to maximise adoption and maximise the positive impact on food and nutrition security throughout the year.

\normalsize
%\vspace*{\fill}
