%\chapter{Summary}
{\Large\textbf{Summary}}
\chaptermark{Summary}
\label{cha:Summary}

Malnutrition affects the quality of life for a substantial portion of the human population. The incidence of adverse health effects due to chronic undernourishment and micronutrient deficiencies are substantially higher in sub-Saharan Africa (SSA) compared to the rest of the world. The individual and societal implications for such nutritional gaps are borne disproportionately by the rural population, indicated by the consistently higher prevalence of stunting in rural SSA. Members of rural households are both vulnerable to the health burdens that stem from food insecurity and central to improving the availability and affordability of a diversity of wholesome foods for both rural and urban markets.

It has been estimated that much of the chronic and hidden hunger that we see today can be alleviated by implementing a suite of nutrition-specific interventions at a cost of US\$9.6 billion per annum. The effectiveness of such investments can be accelerated with complementary food system-based interventions. However, such food system-based interventions are hampered by a limited understanding of the prevalence of food insecurity, the spatial distribution of food insecurity, and the associations between food insecurity and rural livelihoods. The primary objective of this thesis was to describe, analyse and understand food security in rural landholding households in predominantly mixed crop-livestock agricultural systems of sub-Saharan Africa (SSA). The secondary objective was to improve the methodological basis of household level food security studies.

The rural household multi-indicator survey (RHOMIS) tool was developed to describe and analyse the circumstances of rural households -- including food security status. \textbf{Chapter 2} describes the design principles and core modules of the RHOMIS tool. The RHOMIS tool aims to adhere to the principles of being time-efficient, utilitarian, user-friendly, flexible and reliable. The core modules include farm characteristics, household composition, food security, progress out of poverty and `gender'. \textbf{Chapter 2} provides a preliminary evaluation of the time-efficiency and user-friendliness of the RHOMIS tool. The duration of RHOMIS is half that of comparable surveys. The ease of enumeration was evaluated favourably for each module, with over 50\% of interviews being perceived as `easy' by the interviewer.

A critical evaluation of the quality of data produced by rural household surveys is then carried out in \textbf{Chapter 3}. This chapter evaluates the credibility, consistency and reliability of data collected using three different farm household surveys deployed in four African countries. Several limitations in data quality were identified. First, variables which might be considered `easy to estimate' had instances of non-credible observations. For example, in the food security and food self-sufficiency indicators, between 29\% and 57\% of observations in the World Bank's `living standards measurement survey' were deemed beyond credible bounds. In contrast, the shorter and more targeted survey tool, RHOMIS, performed better -- with 25\% of observations beyond credible bounds. Measurements of maize yields and land area owned were found to be less reliable than other variables such as household inhabitants and livestock holdings. This lack of reliability has implications for monitoring food security status, poverty status and the land productivity of households. Despite the limitations in data quality, our analysis shows that if the same farm households are followed over time, the sample sizes needed to detect substantial changes are in the order of hundreds of surveys, and not in the thousands. Chapter 3 concludes that targeted and systematised household surveys are essential in detecting differences and changes in rural communities.

The RHOMIS tool was then used to quantify changes in livelihoods and food security status in an urban linked, high potential region of Tanzania. \textbf{Chapter 4} incorporates paired observations from 20 villages in Lushoto district, Tanzania, over three years. This chapter evaluates the extent to which changes in livelihoods, poverty and food security were taking place in such a short span of time. Households in the study site adaptively responded to local and national circumstances and can be grouped in four clusters, namely: `Rising high value crop', `Rising livestock', `Subsisting mixed' and `Subsisting staples'. A substantial portion of households in the study site made changes to their livelihoods over the short three-year period of observation. Changes in land ownership, livestock-holdings and high value crop production were most likely related to market opportunities and personal circumstances, rather than to direct interventions. Several households made strategic changes by expanding land ownership, planting perennial crops and investing in exotic cattle breeds; many households tactically utilised their land for diversified, mixed crop-livestock production. The majority of households, however, had either remained stable or were scaling back to subsistence farming. \textbf{Chapter 4} finds that the complex risk management strategies and market responsiveness demonstrated by the `Rising' clusters are at odds with single focus activities that external organisations tend to promote.

Subsequently, instances of chronic and hidden hunger were analysed in two subtly contrasting regions of Burkina Faso. \textbf{Chapter 5} describes and analyses the household composition, farm systems and food security status in these regions. Food security indicators were enumerated for two periods to account for the temporal variability throughout the year. Diet diversity (a proxy for hidden hunger) was disaggregated by channel of access to better understand food sourcing behaviour. The results of this chapter show that in both provinces, the ability to purchase food is what differentiates the more food secure households from their less food secure counterparts. This finding does not detract from the utility of subsistence production -- where consumption of own-farm sourced food catered for 91\% of the annual energy requirements in Yatenga province and 72\% in Seno province. Further, households were observed to be pursuing market-oriented strategies in combination with production diversification. Production diversification was not driving differences in diet diversity in this instance but was likely to reduce risk exposure to climatic or economic shocks.

\textbf{Chapter 6} draws on a large sample of rural landholding households across SSA to estimate the prevalence of dietary gaps and to understand their associations with rural livelihoods and food sourcing behaviour throughout the year. As many as 40\% of households were classified as chronically hungry in the lean period. Prevalence of micronutrient dietary gaps were high, ranging from 35\% of households lacking daily sources of pyridoxine, and 68\% lacking daily sources of calcium. Vulnerability to dietary gaps differed by household composition, livelihood characteristics and agro-ecological zone (AEZ). Market participation, livestock product diversity, crop diversity, gross income and off-farm income were all associated with chronic and hidden hunger -- differing by AEZ. Any given livelihood characteristic had a limited impact on food security indicators. Rather, it is the combination of these livelihood characteristics and the agro-ecological production potential that drive the availability of food and income. Households with a livestock component to their farm consumed more milk, meat and eggs (sources of calcium, riboflavin and vitamin B12). Dairy, fruit and vitamin A-rich produce were predominantly accessed through the farm channel -- more so by diverse cropping households. The results of Chapter 6 suggest that households fail to purchase food categories that nutritionally complement their own agricultural products. Furthermore, households with a livestock component to their farm were found to have a lower prevalence of chronic and hidden hunger.

Methodological considerations and empirical findings of this thesis are further evaluated in a general discussion. The general discussion in \textbf{Chapter 7} summarises the progress that has been made in harmonising rural household surveys through the RHOMIS tool. The RHOMIS tool is now available in 9 languages and has been localised to 48 sub-national locations in SSA, South-East Asia and, South and Central America. A community of practice has formed around the use of the tool which provides opportunities for further methodological developments in sampling, validation, alternative modes of data collection, improved enumeration and modelling of women's empowerment and improvements in measuring food and nutrition security.

This thesis has drawn upon a total of 8,257 household interviews, conducted in 14 projects to describe and analyse food and nutrition security in rural landholding households. \textbf{Chapter 7} reiterates the high prevalence rates of chronic and hidden hunger identified in this thesis, noting the variability between and within regions. The factors identified to be associated with food and nutrition security in Chapters 4, 5 and 6 are then analysed and discussed further. In extended analyses, the gender of household head and stage of life were found to be associated with the number of household inhabitants and thus influence nutritional requirements and food security status. The high prevalence of food insecurity, the complexity of associations and the failure to nutritionally complement own-production with purchases have implications for developing effective interventions. The discussion concludes that programs can be designed as `packages' of agricultural and non-agricultural interventions to maximise adoption and maximise the positive impact on food and nutrition security throughout the year.
