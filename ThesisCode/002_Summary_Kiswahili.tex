%\chapter{Summary}
{\Large\textbf{Muhtasari}}
\chaptermark{Muhtasari}
\label{cha:Muhtasari}

Hali duni ya afya inayosababishwa na lishe duni na upungufu wa vurutubisho muhimu ni kubwa sana katika nchi za kusini mwa jangwa la Sahara ikilinganishwa na sehemu nyingine duniani. Athari za madhara binafsi na madhara kwa jamii yanayosababishwa na utofauti huu mkubwa wa kilishe mara nyingi hubebwa na jamii zinazoishi vijijini zikiashiriwa na uwepo mkubwa zaidi wa utapiamlo kwenye mazingira ya vijiji ikilinganishwa na mazingira ya mijini. Jamii zinazoishi kwenye mazingira ya vijijini ziko katika hali hatarishi zaidi ya mizigo ya magonjwa yanayotokana na upungufu wa chakula na tatizo la kuwezesha upanikanaji na uthamani wa mazao mbalimbali ya chakula kwenye masoko ya mijini na vijijini. Inakadiriwa kuwa kiasi kikubwa cha njaa sugu iliyojificha (utapiamlo) tunayoiona wakati huu inaweza kuondolewa kwa kufuata afua zinazolenga lishe bora kwa gharama ya dola bilioni 9.6 kwa mwaka. Matokeo mazuri ya uwekezaji kama huu yanaweza kuongezewa kasi kwa kutumia mfumo wenye hatua zenye kutumia vyakula mbadala ingawa mifumo yenye kutumia afua zinazotumia vyakula mbadala inaharibiwa na uelewa mdogo wa uwepo wa upungufu mkubwa na chakula, mgawanyo wa kutokuwa na ulinzi wa chakula maeneo mengi na uhusiano kati ya uhaba wa chakula na maisha ya vijijini.

Lengo kuu la tafiti hii lilikuwa ni kuelezea, kutathmini na kuelewa uhaba uwepo wa chakula kwenye kaya zilizopo maeneo ya vijijini hasa hasa ambayo huzalisha mazao mchanganyiko, na mifumo ya wakulima wafugaji chini ya jangwa la Sahara. Lengo la pili lilikuwa ni kuboresha misingi ya njia za masomu yanayohusiana na ulinzi wa chakula. 
Nyenzo za ukusanyaji wa takwimu tofauti za tathmini ya viashiria mablimbali vya maisha vijijini zilitengenzwa kuelezea na kutathmini hali halisi za kaya zilizoko maeneo hayo pamoja na hali halisi ya ulinzi wa chakula. 

\textbf{Sura ya 2} inaelezea kanuni na muundo wa msingi wa nyenzo za ukusanyaji wa takwimu hizi za tathmini. Nyenzo za ukusanyaji wa takwimu hizi zina lengo la kuzingatia kanuni za kuwa na ufanisi wa muda, zenye urahisi wa kutumia, kubadilishwa na kuaminika. Vipengele vya msingi vya nyenzo za ukusanyaji wa takwimu hizi ni pamoja na sifa za shamba, muundo wa kaya, lishe, maendeleo ya umaskini na 'jinsia'.

\textbf{Sura ya 3} inatoa tathmini ya awali ya ufanisi wa muda na ni rafiki kwa mtumiaji. Muda wa ambao nyenzo za ukusanyaji wa takwimu hizi hutumia ni takribani nusu ya nyenzo za ukusanyaji wa takwimu za utafiti zinazofanana. Vilevile zaidi ya asilimia hamsini ya mahojiano yameonekana kuwa rahisi kwa mhojiwa. Sura ya 3 inaonyesha maadili yasiyo ya kawaida, thabiti na yenye takwimu za kuaminika zilizokusanywa kwa kwenye tafiti tatu za kilimo kwa ngazi ya kaya katika nchi nne za Afrika. Vikwazo kadhaa vya ubora wa takwimu vilibainishwa. Kwanza, majibu ambayo watafiti wanafikiri ni 'rahisi kupatikana' yalikuwa na matukio ya makosa. Kwa mfano, viashiria vya lishe na upatikanaji wa chakula cha kutosha, kati ya asilimia 29 na 57 ya tathmini zilizofanyika za hali ya ubora wa maisha kwa `viwango vya maisha' vya bank kuu ya dunia zilionekana kuwa na makosa ikilinganishwa na nyenzo za kukusanya takwimu za mfupi na ambazo huchukua walengwa zaidi, (RHOMIS) zilifanya vizuri kwa asilimia zaidi 25\% ya tathmini.
Vilevile, upimaji wa mavuno ya mahindi, na ukubwa wa mashamba ulionekana kuwa hauna uhakika zaidi ikilinganishwa na vigezo vingine kama vile idadi ya watu na umiliki wa mifugo. Ukosefu wa kuaminika wa nyenzo za ukusanyaji wa takwimu hauthiri tathmini za ukosefu wa lishe, hali ya umasikini na mavuno ya mazao kwa kaya. Licha ya mapungufu katika ubora wa takwimu, uchambuzi wetu unaonyesha kuwa ikiwa kaya zinafuatiliwa, ukubwa wa sampuli unaohitajika kutoa matokeo sahihi sio wa kaya maelfu bali ni wa kaya mamia. Sura ya 3 inahitimisha kuwa, uchunguzi wa kaya unaofuata utaratibu sahihi ni muhimu sana katika huhakikisha kuwa kupata tawimu sahihi za tofauti na mabadiliko katika jamii zilizoko vijijini.

Nyenzo za ukusanyaji wa takwimu za RHOMIS zilikuwa kilikuwa zinatumiwa kupima mabadiliko ya hali ya maisha na hali ya lishe nchini Tanzania katika nyanja zinazohusiana na masoko na zina uwezo mzuri wa uzalishaji wa takwimu sahihi. \textbf{Sura ya 4} inafafanua uchunguzi uliofanyika kwenye vijiji 20 vilivyoainishwa katika wilaya ya Lushoto, Tanzania kwa zaidi ya miaka mitatu.
Sura hii inachunguza kiwango cha mabadiliko ya maisha, umasikini na lishe kilichotokea kwa kipindi cha muda mfupi majumbani. Tafiti za majumbani zilizofanyika kwenye tovuti hutegemea mazingira ya ndani ya nchi husika ambayo yanaweza kugawanywa katika makundi manne, yaani: `kuongezeka kwa thamani ya mazao', `kuongezeka kwa thamani ya mifugo', `mazao mchanganyiko ya kujikimu' and `mazao makuu ya kujikimu'. Sehemu kubwa ya kaya katika tovuti ya utafiti zilifanya mabadiliko ya maisha yao kwa muda mfupi wa uchunguzi wa miaka mitatu.
Mabadiliko katika umiliki wa ardhi, ufugaji wa mifugo na uzalishaji wa mazao ya thamani ya juu yalikuwa na uwezekano mkubwa wa kuhusiswa na fursa za masoko na hali za kibinafsi, badala ya hali za kijamii. Kaya kadhaa zilifanya mabadiliko ya kimkakati katika kupanua umiliki wa ardhi, kuwekeza katika mifugo na kukuza mazao ya thamani ya juu; kaya nyingi zilitumia ardhi yao kwa ajili ya kilimo cha mazao mbalimbali, na uzalishaji wa mifugo. hata hivyo, kaya nyingi, zilikuwa imara au zilikuwa zikitumia mazao yaliyozalishawa nyumbani. Sura ya 4 inabainisha kuwa mikakati ya usimamizi na ufumbuzi wa masoko imewezeshwa sana na makundi ya vyakula 'yanayochipukia' ikilinganishwa na kazi za ambazo zinahimizwa sana na mashirika ya nje. 

Vilevile, matukio ya `ukosefu wa nishati' na `ukosefu wa virutubishi' yalichambuliwa katika mikoa miwili tofauti ya Burkina Faso. \textbf{Sura ya 5} inaelezea na inachambua muundo wa kaya, mifumo ya kilimo na hali ya lishe katika mikoa hii. Viashiria vya lishe vilichapishwa kwa muda wa vipindi viwili kwenye kwa utofauti wa mwaka mzima. Njia za upatikanaji wa chakula mchanganyiko (kipimo cha `ukosefu ya virutubishi') zilitofautiana. Matokeo ya sura hii yanaonyesha tofauti mablimabali kujipatia chakula ikilinganishwa na uwezo wa kipato wa kaya kwenye mikoa yote. Matokeo haya hayakupunguza thamani ya chakula kinachotumiwa kutoka mashambani ambacho huchangia asilimia 91 ya mahitaji mwaka ya nishati katika jimbo la Yatenga na asilimia 72 katika jimbo la Seno. Zaidi ya hayo, kaya zilionekana kuzingatia mikakati inayotokana na soko, pamoja na aina mablimbali za uzalishaji. Hivyo basi, uzalishaji mbadala huenda ukapunguza hatari ya hali ya hewa au hatari ya kiuchumi.

\textbf{Sura ya 6} inatathmini sampuli kubwa ya kaya za vijijini nchini Afrika ya Kusini ili kukadiria upungufu ya chakula na kuelewa vyanzo vya kipato na maisha ya vijijini na upatikanaji wa chakula kwa mwaka mzima. Takriban asilimia 38 ya kaya zilionekana kuwa na `njaa kali'. Hali ya kuenea kwa upungufu wa chakula ulikuwa ni ya juu, kutoka takribani asilimia 41 ya kaya kukosa vyanzo vya chakula chenye madini chuma kila siku na madini ya `thiamine' na asilimia 72 kutokuwa na vyanzo vya kila siku vya vitamini B12. Mapungufu ya chakula hutofautiana na mfumo wa kaya sifa ya sifa nyanda za ki-agroikolojia. Aina za masoko, utofauti wa mifugo, utofauti wa mazao, mapato ya jumla na mapato ya nje ya kilimo yalihusishwa na `upungufu wa virutubishi' ikilinganishwa na nyanda za ki-agroikolojia. Tabia moja pekee ilikuwa na athari ndogo juu ya viashiria vya usalama wa chakula. Badala yake, mchanganyiko wa tabia hizi pamoja na eneo la ki-agroekolojia huchangia mabadiliko katika mapato na upatikanaji wa chakula. Kaya zilizo kuwa na mifugo zilitumia maziwa zaidi, nyama na mayai (vyanzo vya madini chuma, riboflavin na vitamini B12) Maziwa, matunda na vitamini A, hupatikana kwa wingi kadri mazao ya mashamba yanavyoongezeka. Matokeo ya Sura ya 6 yanaonyesha kwamba kaya hazijinunulii makundi ya chakula ili kukidhi mahitaji yao yote ya lishe. Zaidi ya hayo, kaya zilizo na mifugo zilikuwa na lishe bora kuliko nyingine.

Masuala ya njia ya utafiti na matokeo ya utafiti huu yanatathminiwa zaidi katika mjadala wa jumla. Mjadala wa jumla katika \textbf{Sura ya 7} unaonesha muhtasari wa maendeleo ambayo yamefanyika katika kuunganisha tafiti za kaya za vijijini kwa njia ya nyenzo za ukusanyaji wa takwimu za RHOMIS. Nyenzo za ukusanyaji wa takwimu za RHOMIS sasa zinapatikana katika lugha 9 na zimefanyika maeneo 48 ya kitaifa katika nchini Afrika ya kusini, Kusini-mashariki mwa Asia na, kusini na katikati mwa Marekani. Kwa kuwa kuna watumiaji wengi wenye mahitaji ya nyenzo hii, kuna fursa za maendeleo zaidi ya ya nyenzo hii ikiwa ni pamoja na sampuli, uthibitisho, njia mbadala za ukusanyaji wa takwimu, kuelewa uwezo wa wanawake na kuboresha upimaji lishe. Tafiti hii imefanya jumla ya mahojiano ya kaya 8,257, katika miradi 14 inayolenga kuelezea na kuchambua masuala ya lishe ya kaya vijijini. 

\textbf{Sura ya 7} inaelezea viwango vya juu vya kuenea kwa `upungufu wa nishati' na `upungufu wa virutubishi', akibainisha tofauti kati ya mikoa na maeneo ndani ya nchi husika. Sababu ambazo zinahusiana na lishe katika Sura ya 4, 5 na 6 zimeonesha kufuatiliwa na kujadiliwa zaidi. Uchambuzi zaidi umeonesh kuwa jinsia, ukuu wa kaya na hali ya maisha vimehusishwa zaidi na idadi ya watu kwenye kaya na basi kuathiri mahitaji ya lishe na hali ya `kukosa lishe'. Hiyo ni mchanganyiko wa sifa hizi za maisha ambazo huamua hali ya usalama wa chakula kwa mwaka mzima. Matokeo haya yote yana maana katika kuendeleza hatua za ufanisi. Mjadala huu unahitimisha kuwa programu zinaweza kuundwa kama 'vikundi' ya hatua za kilimo na zisizo za kilimo ili kuboresha hatua za kupunguza lishe kwa mwaka mzima.
