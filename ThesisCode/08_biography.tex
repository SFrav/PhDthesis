%\chapter{About the author} 
{\Large\textbf{About the author}}
\chaptermark{About the author}

\small
Simon Fraval was born on the 30th of December 1983 in Ascot, England. In 2007, Simon earned a Bachelor of Commerce with honors in economics from Deakin University (Melbourne, Australia). His minor thesis was titled `Determining financial support for Australian electricity generators in an emission constrained economy'. 

Between 2008 and 2012, Simon was employed by the state government of Victoria, Australia. During his tenure, he was stationed at the head office in Melbourne, the Rutherglen lamb and viticulture research centre, and the national centre for dairy research and development. His research interests centred on agricultural trade, including topics of production trends, market access, animal welfare of live exports and the environmental impacts of exported livestock products. In 2010, Simon co-authored an article published in the Journal of Dairy Science, titled \textit{`The water footprint of dairy products: case study involving skim milk powder'}. In 2012, Simon presented his work titled \textit{`Sustainable Value Chain Analysis: Victorian lamb exported to the USA'} to the IFAMR conference in Frankfurt, Germany. 

In 2012, Simon joined the International Livestock Research Institute (ILRI). Soon after moving to Nairobi, Kenya, Simon took part in a Comprehensive Africa Agriculture Development Programme (CAADP) meeting, discussing the alignment between policy and research agendas. He then focused his research on the quantification and mitigation of greenhouse gas (GHG) emissions in smallholder diary production, with research sites in western Kenya. In collaboration with the FAO, the Kenyan ministry of agriculture and Unique Forestry, Simon co-authored a methodology for incentivising GHG mitigation, titled \textit{`Smallholder dairy methodology: Draft Methodology for Quantification of GHG Emission Reductions from Improved Management in Smallholder Dairy Production Systems using a Standardized Baseline'}. 

Simon continued his research in ILRI under the auspice of the project titled `Comprehensive Livestock Environmental Assessment for Improved Nutrition, a Secured Environment and Sustainable Development' (CLEANED). In a collaboration between ILRI, CIAT, SEI and CSIRO, the CLEANED project sought to provide a method to estimate the environmental impact of livestock interventions in low and middle income countries prior to implementation. Simon was involved throughout the research process, from conceptual development, piloting, analysis and reporting. Simon led the GHG and biodiversity components of the framework and co-led participatory GIS activities in Kenya and Tanzania.    

In 2015, Simon began the field work for his PhD. Hosted by ILRI and the Animal Production Systems group of Wageningen University \& Research, Simon sought to understand and describe the nature of food security in rural sub-Saharan Africa. During the process of his PhD, Simon co-authored five peer-reviewed articles and two book chapters. Simon has also been active in the research community -- reviewing journal articles and presenting his work.
Simon's research interests include topics related to food security, rural livelihoods, environmental impact assessment and remote sensing. 

%\lipsum[1-3]


\newpage

\small
{\normalsize\textbf{Peer-reviewed journal publications}}
%\vspace{-0.5cm}


Ridoutt, B. G., Williams, S. R. O., Baud, S., Fraval, S., \& Marks, N. (2010). Short communication: The water footprint of dairy products: case study involving skim milk powder. \textit{Journal of Dairy Science}, 93(11), 5114–7. doi:10.3168/jds.2010-3546

Hammond, J., Fraval, S., van Etten, J., Suchini, J. G., Mercado, L., Pagella, T., Frelat, R., Lannerstad, M., Douxchamps, D., Teufel, N., Valbuena, D., \& van Wijk, M. T. (2017). The Rural Household Multi-Indicator Survey (RHoMIS) for rapid characterisation of households to inform climate smart agriculture interventions: Description and applications in East Africa and Central America. \textit{Agricultural Systems, 151}, 225–233. doi:10.1016/j.agsy.2016.05.003

Ng'ang'a, S., Ritho, C., Herrero, M., \& Fraval, S. (2018). Household-oriented benefits largely outweigh commercial benefits derived from cattle in Mabalane District, Mozambique. \textit{The Rangeland Journal, 40(6)}, 565–576. doi:10.1071/RJ17115

Tavenner, K., Fraval, S., Omondi, I., \& Crane, T. A. (2018). Gendered reporting of household dynamics in the Kenyan dairy sector: trends and implications for low emissions dairy development. Gender, Technology and Development, 22(1), 1–19. doi:10.1080/09718524.2018.1449488

Fraval, S., Hammond, J., Wichern, J., Oosting, S. J., de Boer, I. J. M., Teufel, N., Lannerstad, M., Waha, K., Pagella, T., Rosenstock, T. S., Giller, K. E., Herrero, M., Harris, D., \& van Wijk, M. T. (2018). Making the most of imperfect data: a critical evaluation of standard information collected in farm household surveys. \textit{Experimental Agriculture}, 1–21. doi:10.1017/S0014479718000388.

Fraval, S., Hammond, J., Lannerstad, M., Oosting, S. J., Sayula, G., Teufel, N., Silvestri, S., Poole, E. J., Herrero, M., \& van Wijk, M. T. (2018). Livelihoods and food security in an urban linked, high potential region of Tanzania: Changes over a three year period. \textit{Agricultural Systems, 160 (January 2018)}, 87–95. doi:10.1016/j.agsy.2017.10.013

{\normalsize\textbf{Submitted manuscripts}}

%\vspace{-0.5cm}
Fraval, S., Yameogo, V., Ayantunde, A., Hammond, J., de Boer, I. J. M., Oosting, S. J., \& van Wijk, M. T. (submitted). Pathways to food security in rural Burkina Faso. \textit{Food Security}

Fraval, S., Hammond, J., Bogard, J. R., Ng'endo, M., van Etten, J., Herrero, M., Oosting, S. J., de Boer, I. J. M., Lannerstad, M., Teufel, N., Lamanna, C., Rosenstock, T. S., Pagella, T., Vanlauwe, B., Dontsop-Nguezet, P. M., Baines, D., Carpena, P., Njingulula, P., Okafor, C., Wichern, J., Ayantunde, A., Bosire, C., Chesterman, S., Kihoro, E., Rao, J., Skirrow, T., Steinke, J., Stirling, C. M., Yameogo, V., \& van Wijk, M. T. (submitted). Dietary gaps in sub-Saharan Africa: prevalence and implications for agricultural interventions. \textit{Global Food Security}

Ritzema, R., van Wijk, M., Phengsavanh, P., Hammond, J., Fraval, S., Hok, L., Thi Minh Long, C., Bolliger, A., \& Douxchamps, S. (submitted). Household-level drivers of dietary diversity in transitioning agricultural systems: Evidence from the Greater Mekong Subregion
Stanley northern Kenya. \textit{Agricultural systems}



{\normalsize\textbf{Book chapters}}

%\vspace{-0.5cm}
Fraval, S., van Middelaar, C., Ridoutt, B., \& Opio, C. (2018). Life Cycle Assessment of Food Products. In P. Ferranti, E. M. Berry, \& J. R. Anderson (Eds.), \textit{Encyclopedia of Food Security and Sustainability} (Vol. 3, pp. 488–496). Elsevier. doi:10.1016/B978-0-08-100596-5.21427-3

Wijk, M. T. Van, Hammond, J., Frelat, R., \& Fraval, S. (2018). Unequal Access to Land: Consequences for the Food Security of Smallholder Farmers in Sub Saharan Africa. In P. Ferranti, E. M. Berry, \& J. R. Anderson (Eds.), \textit{Encyclopedia of Food Security and Sustainability} (pp. 1–6). Elsevier. doi:10.1016/B978-0-12-812687-5.22311-3



{\normalsize\textbf{Other scientific publications}}

%\vspace{-0.5cm}
Opio, C., Gerber, P., Baltenweck, I., Fraval, S., Mbae, R., Kessei, L., Wilkes, A., Tennigkeit, T., Marroquin, L., Baumann, T., \& Hardy, P. (2016). \textit{Smallholder dairy methodology: Draft Methodology for Quantification of GHG Emission Reductions from Improved Management in Smallholder Dairy Production Systems using a Standardized Baseline}.

Morris, J., Fraval, S., Githoro, E., Ran, Y., \& Mugatha, S. (2015). \textit{Comprehensive Livestock Environmental Assessment for Improved Nutrition, a Secured Environment and Sustainable Development along Livestock and Aquaculture Value Chains Project: PGIS workshop summary report}. (2015 No. 05). Stockholm.

{\normalsize\textbf{Conference proceedings}}

%\vspace{-0.5cm}
Fraval, S., Marks, N., Fearne, A., \& Ridoutt, B. G. (2011). Sustainable Value Chain Analysis: Victorian  lamb exported  to  the USA. \textit{The Road to 2050: Sustainability as a Business Opportunity}. Frankfurk: IFAMR. Available on \url{https://www.ifama.org/2011-Frankfurt}

Fraval, S., Hammond, J., Bogard, J. R., Ng'endo, M., van Etten, J., Herrero, M., Oosting, S. J., de Boer, I. J. M., Lannerstad, M., Teufel, N., Lamanna, C., Rosenstock, T. S., Pagella, T., Vanlauwe, B., Dontsop-Nguezet, P. M., Baines, D., Carpena, P., Njingulula, P., Okafor, C., Wichern, J., Ayantunde, A., Bosire, C., Chesterman, S., Kihoro, E., Rao, J., Skirrow, T., Steinke, J., Stirling, C. M., Yameogo, V., \& van Wijk, M. T. (2018). Food access channels in tropical sub-Saharan Africa. In \textit{Accelerating the End of Hunger and Malnutrition}. Bangkok: IFPRI-FAO. Available on \url{https://www.ifpri-faobangkokconference.org/files/2018/11/Slide43.jpg}

Fraval, S., Hammond, J., Bogard, J. R., Ng'endo, M., van Etten, J., Herrero, M., Oosting, S. J., de Boer, I. J. M., Lannerstad, M., Teufel, N., Lamanna, C., Rosenstock, T. S., Pagella, T., Vanlauwe, B., Dontsop-Nguezet, P. M., Baines, D., Carpena, P., Njingulula, P., Okafor, C., Wichern, J., Ayantunde, A., Bosire, C., Chesterman, S., Kihoro, E., Rao, J., Skirrow, T., Steinke, J., Stirling, C. M., Yameogo, V., \& van Wijk, M. T. (2019). Dietary gaps in rural sub-Saharan Africa. \textit{4th International Congress on Hidden Hunger}. Stuttgart: SNFS. Available on \url{http://hiddenhunger.net}

Teufel, N., Hammond, J., Fraval, S., \& van Wijk, M. T. (2018). Dynamics of food security and livelihood strategies in Eastern Africa. In \textit{Strengthening food and nutrition security through a Planetary Health lens in resource-limiting settings conference}. Dar es Salaam: University of Sydney. Available on \url{https://sydney.edu.au/vetscience/research/Nkuku4U/images/Nkuku4U-program.pdf}
\normalsize
